\documentclass[9pt,oneside]{amsart}
\usepackage{url}
\usepackage{cancel}
\usepackage{xspace}
\usepackage{graphicx}
\usepackage{multicol}
\usepackage{subfig}
\usepackage{amsmath}
\usepackage{amssymb}
\usepackage[a4paper,width=170mm,top=18mm,bottom=22mm,includeheadfoot]{geometry}
\usepackage{booktabs}
\usepackage{array}
\usepackage{verbatim}
\usepackage{caption}
\usepackage{natbib}
\usepackage{float}
\usepackage{pdflscape}
\usepackage{mathtools}
\usepackage[usenames,dvipsnames]{xcolor}
\usepackage{afterpage}
\usepackage{tikz}
\usepackage{csquotes}
\usepackage[english,french,german]{babel}
\usepackage[utf8]{inputenc}
\usepackage{lipsum}

\newcommand{\hcancel}[1]{
    \tikz[baseline=(tocancel.base)]{
        \node[inner sep=0pt,outer sep=0pt] (tocancel) {#1};
        \draw[black] (tocancel.south west) -- (tocancel.north east);
    }
}

\definecolor{lightyellow}{rgb}{1,0.98,0.9}
\definecolor{lightblue}{rgb}{0.9,0.9,0.98}
\definecolor{lightpink}{rgb}{1,0.94,0.95}

\definecolor{initial}{rgb}{.7,0.0,0.0}
\definecolor{reviewed}{rgb}{0.0,0.0,0.7}

\providecommand{\todo}[1]{{\color{initial}#1}}
\providecommand{\review}[1]{{\color{reviewed}#1}}

\newcommand{\firsthomesteadblock}{\ensuremath{\mathit{TBA}}}

\DeclarePairedDelimiter{\ceil}{\lceil}{\rceil}
\newcommand*\eg{e.g.\@\xspace}
\newcommand*\Eg{e.g.\@\xspace}
\newcommand*\ie{i.e.\@\xspace}

\title{Votesapp: On Issues Arising With Blockchain Voting \\ {\smaller \textbf{Position Paper, Early Draft}, \today}}
\author{
    Afri Schoedon\\
    Co-Founder \& Blockchain Architect, Votesapp\\
    afri@votesapp.net
}
\begin{document}

\pagecolor{lightblue}

\begin{abstract}
\end{abstract}

\maketitle

\setlength{\columnsep}{20pt}
\begin{multicols}{2}

\section{Motivation}\label{sec:motivation}

% decentralize democratic processes [VARSHNEYA~et~al.~2015]
% promote election fairness [VARSHNEYA~et~al.~2015]
% promote online voting security [VARSHNEYA~et~al.~2015]
% motivation: electronic voting machines: hardware + software attacks [VARSHNEYA~et~al.~2015]
% motivation: voting authority centralization [VARSHNEYA~et~al.~2015]
% motivation: trusted central auditing authority [VARSHNEYA~et~al.~2015]
% motivation: increase security [VARSHNEYA~et~al.~2015]

\section{On blockchain voting}

\subsection{Why blockchain matters}

\subsection{Blockchain voting vs votechains}
Voting is a key functionality of blockchain technology. The underlying consensus protocols on most distritubted peer-to-peer solutions always have to include ways to update protocols and restore consensus upon disputes. A prominent example is the Bitcoin Improvement Proposal \#BIP34, of the infamous crypto-currency Bitcoin [NAKAMOTO~2008], proposing miner behaviour on softforking mechanics.\par
This is voting on a very technical level and not considered part of this paper.\par
The following projects and voting references rather focus on applications on top of blockchains, enabling democratic participation in any kind of groups or societies.

\subsection{Projects utilizing blockchain voting}
Below is a short overview of existing research, organizations and implementations of blockchain based democracy applications.

\subsubsection{Follow My Vote}
Follow My Vote, formerly \enquote*{Vote DAC} [ERNEST~2014] is a distributed autonomous company (DAC) emerged from the Bitshares community. It's currenlty heavily developing a stake-weighted blockchain voting DAC by a team around Adam Kaleb Ernest [ERNEST~2016].
\subsubsection{Public Votes}
Public Votes is an Ethereum voting application designed and implemented by Dominik Schiener. Both his proposal [SCHIENER~2015A] and more recent analytics[SCHIENER~2015B] are highlighting advantages and issues arising with Ethereum and blockchain voting.
\subsubsection{Bitmessage Vote}
In 2014 Jesper Borgstrup wrote a master thesis on \enquote*{Private, trustless and decentralized message consensus and voting schemes} [BORGSTRUP~2014]. The protocol uses blockchain-technology and invertible bloom lookup tables for defining deadlines and timestamping of messages. Linkable ring signatures provide a scheme suitable for signing votes. A reference implementatoin is based on the PyBitmessage applicatoin.
\subsubsection{BitVote}
BitVote is an Ethereum decentralized application (DApp) using encryption chains and a peer-to-peer hybrid technology for the purpose of proposal collaboration, information sharing and voting. There is a whitepaper draft available [OUDEN~2014].
\subsubsection{Nemos.io}
Developed by France's netparty, Nemos.io is a blockchain-proofed decision making tool based on Ethereum with eleminated gas costs.
\subsubsection{Blockchain Apparatus}
The Blockchain Apparatus aims to become a blockchain-secured voting machine.
\subsubsection{Quadratic Voting $(V)^2$}
$(V)^2$ emerged from the Ether.camp hackathon in december 2015 and developed a quadratic voting dapp based on the Ethereum network.
\subsubsection{Democracy Earth}
Democracy.earth is a follow-up project by Santiago Siri and Pia Mancini from DemocracyOS who relocated to the United States recently. The project is currently creating a community, recruiting enthusiasts and researching on blockchain voting and identity. No possible solutions have been published yet [MANCINI~et~al.~2015].\par
DemocracyOS is a centralized web-application from Argentina which allows you to propose, debate and vote online. The team discussed blockchain integration back in 2015 twice [DEMOCRACYOS~2015A][DEMOCRACYOS~2015B]. An article on Newschallange contains first mockups which appear to use the Bitcoin blockchain for simple voting transactions [MANCINI~2015].
\subsubsection{SureVoting}
Students from the West Virginia University WVU, Ricky Kirkendall and Ankur Kumar, are currently creating SureVoting, a student government voting app for iPads. There is no concept or code released yet though [COYNE~2015].
\subsubsection{VoteCoin}
Votecoin is a proposal for a hash based voting technology. The whitepaper calls for a fair, transparent, practical solution but fails to deliver details on implementation [LEVEL~2014].
\subsubsection{Agora Voting}
AgoraVoting had the idea to add a distributed voting system on top their working solutions [ELVIRA~2013] but failed to raise the required development funds.
\subsubsection{BitCongress}
BitCongress claims to be a decentralized voting platform but is lacking references or source code, a whitepaper is available though [ROCKWELL~2014].
\subsubsection{V-Initiative}
V-Initiative proposed a decentralized voting app but the project seems dead already.

\subsection{Scope for Votesapp}
The team for Votesapp wants to provide a blueprint for a better democratic process and develop blockchain based tools which educate, inform, support the decision making. This paper is initialized by them and tries to connect the projects above by promoting an improved cross-initiatives collaboration.

\section{General outline on key issues}

\subsection{Architecture considerations}

\subsection{Blockchain voting scalability}

\subsection{Transaction fees issue}

\subsection{Blockchain identity issue}

\subsection{Decentralized deadline consensus problem}

\section{Outlook}

\section{License}
This paper is available in the public domain (CC0).

\section{References}
[BORGSTRUP~2014] Borgstrup, J (2014): Private, Trustless And Decentralized Message Consensus And Voting Schemes, Master's Thesis, 23 November~2014, University of Copenhagen, Denmark.\par
[BUTERIN~2014] Buterin, V (2014): A Next-Generation Smart Contract And Decentralized Application Platform, Ethereum Whitepaper, 11 January~2014, Ethereum Project: http://vbuterin.com/ethereum.html\par
[COYNE~2015] Cyone, C (2015): New Voting System Proposed For 2016 SGA Elections, The Daily Athenaeum, 19 November 2015, Morgantown, West Virginia, United States of America.\par
[DEMOCRACYOS~2015A] Anonymous (2015): Blockchain support for DemocracyOS, DemocracyOS Blog, 19 March 2015, Buenos Aires, Argentina.\par
[DEMOCRACYOS~2015B] Anonymous (2015): Why blockchain support for DemocracyOS, DemocracyOS Blog, 24 June 2015, Buenos Aires, Argentina.\par
[ELVIRA~2013] Elvira, ER (2013): A Bitcoin Based, Completely Distributed Voting System, 28 November 2013, Agora Voting Blog, Madrid, Spain.\par
[ÉPIÉ~et~al.~2015] Épié, C; Loubet, N (2015): Blockchain and Beyond, Version 1.0, 3 December~2015, Cellabz, Paris, France.\par
[ERNEST~2014] Ernest, AK (2014): The Key To Unlocking Teh Black Box: Why The World Needs A Transparent Voting DAC, 04 July~2014, Follow My Vote, Blacksburg, Virginia, United States of America.\par
[ERNEST~2016] Ernest, AK (2016): Follow My Vote, 26 January~2016, In: Report On Secure Voting, A Guide To Secure \#onlinevoting In Elections, Webroots Democracy, United Kingdom.\par
[HOURT~2015A] Hourt, N (2015): End to End Verified Voting On A Blockchain, 3 September~2015, Follow My Vote, Youtube: https://youtu.be/EE2mWoio7po\par
[HOURT~2015B] Hourt, N (2015): How Our Voter Registration Process Works, 16 December~2015, Follow My Vote, Youtube: https://youtu.be/GcAz9mZW1\_c\par
[JOHNSTON~et~al.~2013] Johnston, DA; Yilmas, SO; Kandah, J; Bentenitis, N; Hashemi, F; Gross, R; Wilkinson S; Mason, S (2013): The Generalized Theory Of Decentralized Applications, DApps,~20 November~2013, DAppsFund, Austin, Texas, United States of America.\par
[KAYE~2014] Kaye, M (2014): A First Attempt To Describe The Neutral Voting Bloc, 2 July~2014, FluxVote, Sydney, Australia.\par
[KONASHEVYCH~2016] Konashevych, O (2016): What Are We Doing, 25 February~2016, E-Vox.org, Kiev, Ukraine.\par
[LARIMER~2016] Larimer, D (2016): How To Build A Decentralized Application Without Fees, 10 February~2016, Bitshares, Blacksburg, Virginia, United States of America.\par
[LEVEL~2014] Level, C (2014): VoteCoin, A fair, transparent, practical solution to fix modern voting, VoteCoin Blog, 9 December 2014, United States of America.\par
[MANCINI~2015] Mancini, P (2015): Blockchain Support For Open Source Platform Democracyos, Newschallenge, 17 March 2015, San Francisco, California, United States of America.\par
[MANCINI~et~al.~2015] Mancini, P; Siri, S (2015): Democracy Earth, Power In Your Hands, A Decentralized Global Commons Of Peers, San Francisco, California, United States of America.\par
[MONEGRO~2014] Monegro, J (2014): The Blockchain Application Stack, 25 November~2014, Union Square Ventures, New York, United States of America.\par
[NAKAMOTO~2008] Nakamoto, S (2008): Bitcoin, A Peer-to-Peer Electronic Cash System, Bitcoin Whitepaper, 31 October~2008, Metzdowd Cryptography Mailing List: https://bitcoin.org/bitcoin.pdf\par
[NOIZAT~2014] Noizat, P (2014): Blockchain Electronic Vote, Unchain.Voting Whitepaper, 30 July~2014, Paymium, Paris, France.\par
[OUDEN~2014] Ouden, J (2014): BitVote, \enquote{A-Plan}, 11 June 2014, Eindhoven, Netherlands.\par
[ROCKWELL~2014] Rockcoons \enquote{Rockwell}, M (2014): BitCongress, Whitepaper, Blockchain Based Voting System, 16 October~2014, Bitcoin Kinetics, Beaverton, Oregon, United States of America.\par
[SCHIENER~2015A] Schiener, D (2015): Public Votes, Ethereum-based Voting Application, 28 October~2015, South Tyrol, Italy.\par
[SCHIENER~2015B] Schiener, D (2015): Voting on the Ethereum Blockchain, An Analysis, 7 November~2015, South Tyrol, Italy.\par
[SCOTT~2016] Scott, B (2016): How Can Cryptocurrency And Blockchain Technology Play A Role In Building Social And Solidarity Finance, Working Paper~2016-1, 25 February~2016, United Nations Research Institute for Social Development, Geneva, Switzerland.\par
[VARSHNEYA~et~al.~2015] Varshneye, AJ; Poudel, S; Vyas, X (2015): Blockchain Voting, CS4501 Cryptocurrency Cabal, 07 December~2015, University of Virginia, United States of America.\par
[WOOD~2014] Wood, G (2014): Ethereum, A Secure Decentralised Generalised Transaction Ledger, Ethereum Yellowpaper, 6 April~2014, Ethereum Project: http://gavwood.com/paper.pdf


\end{multicols}

\end{document}
