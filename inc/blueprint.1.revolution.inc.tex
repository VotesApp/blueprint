
\section{Digital Revolution}\label{sec:motivation}

Democracies are built on their population’s consent, and a trustworthy voting system is crucial to this consent. [\cite{Naveen05}]\par

Known electronic voting systems are usually proprietary and centralized, e.g., a single entity controls everything: the hardware, the code base, the data base, system outputs and even the monitoring tools [\cite{NOIZAT2014}]. Adam Kaleb Ernest of the \textsc{Follow My Vote} project describes the electronic voting process in the United States as \textit{black box voting} and links the lack of transparency directly to lower voter turnouts [\cite{ERNEST2014}]. In addition, the used systems have been vulnerable for software and hardware attacks for a long time. [\cite{VARSHNEYA2015}].\par

Chaum and Neff have proposed cryptographic schemes that work without the need to trust the integrity of any software in a voting system. These protocols come with multiple advantages: voters can verify that their votes have been accurately recorded, and everyone can verify that the tallying procedure is correct, preserving privacy and coercion resistance in the
process. The ability for anyone to verify that votes are counted correctly is particularly exciting, as no prior system has offered this feature. Yet,  several potential weaknesses in these voting protocols have been discovered which only became apparent when considered in the context of an entire voting system. [\cite{Naveen05}] \par

A number of alternative proposals have originated from the realm of Blockchain systems. Going beyond the pure execution of polls, they often pose as a means of decentralizing the democratic process to promote election fairness and security. Yet, the idea of a complete decentralization is infeasible considering identity confirmation must occur in any official setting. Verifying users’ credentials requires a central authority in order to avoid Sybil attacks on the system. [\cite{VARSHNEYA2015}] \par

In this paper we want to pick up on the cryptographic protocols of Chaum and Neff and see how they can be combined with blockchain technology to use the best of both worlds. We want to design a system that makes polling and collaborative decision making more accessible while keeping the good properties of verifiability, preserving privacy and coercion resistance.\par